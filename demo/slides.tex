%% -*- latex -*-
\newtoks\nicsroot\directlua{ tex.settoks("nicsroot", os.getenv("NICS_ROOT") or error("NICS_ROOT environment variable has to be set, use the Makefile")) }
\input{\the\nicsroot /src/nics-cached.tex}
\endofdump
\input{\the\nicsroot /src/nics-noncached.tex}
\hypersetup{
  pdfauthor={your name or emailss},
  pdftitle={your presentation title},
}
\nicsgrid=0
\begin{document}

\section{Hello world}

\nicstitleslide{images/welcome}{nics cheetsheat}{Look at the PDF and look at the code}

\begin{slide}{Hello world}{This is our first slide}
  \begin{nicscolumn}
    \nicsheader{Welcome everyone}
    \nicspar{We can have some paragraphs without bullet points}
    \nicsitem{Or with bullet points}
    \begin{nicsindent}
      \nicsitem{\just
        We can also have indentation for less important stuff inside another item.
        Also note that this paragraph is justified and has the same length in each line.
        This is pretty nice for long paragraphs, we have the same in books.
      }
      \nicsitem{\nicsleadword{Lead word} is something that changes the indentation for the rest of the item or paragraph.
        Here the ``Lead word'' at the beginning is defined as lead word.
      }
    \end{nicsindent}
  \end{nicscolumn}
  \begin{nicscolumn*}{5cm}{0cm}{5cm}
    \nicsitem{Absolute positioning is easy with nics}
    \nicsitem{\nicslonghbox{You are bound only by your imagination and the papersize of course}}
    \nicsmath
    \nicsitem{$ e^{i\pi} = -1 $ ... WOW!}
  \end{nicscolumn*}
\end{slide}

\begin{slide}{Slide with title, but without sub title}{}
  \begin{nicscolumn}
    \nicsitem{\mono{Typewriters were awesome, but loud!}}
    \nicsitem{\serif{Serif fonts are good for books, maybe not for presentations}}
    \nicsitem{\sans{You can be \slant{explicit}, but sans serif fonts are the \bold{default}}}
    \nicshrule
    \nicspar{\centering centering is easy}
    \nicspar{\centering\nicsincludepic[width=3cm]{images/tux}}
  \end{nicscolumn}
\end{slide}

\begin{slide}{Skips between paragraphs/items}{}
  \begin{nicscolumn}
    \nicsitem{Here will be a small skip}
    \nicssmallskip
    \nicsitem{And then after this a medium skip}
    \nicsmedskip
    \nicsitem{And finally a big skip}
    \nicsbigskip
    \nicsitem{Should we also have huge skips?  Maybe!}
  \end{nicscolumn}
\end{slide}

\begin{slide}{Skips can be automated}{So you don't have to repeat them between every item}
  \begin{nicscolumn*}{1cm}{2cm}{2cm}
    \nicsheader{\tiny\mono{\bs nicsnoskips}}
    \nicsnoskips
    \nicsitem{point 1}
    \nicsitem{point 2}
    \nicsitem{point 3}
    \nicsitem{point 4}
  \end{nicscolumn*}
  \begin{nicscolumn*}{4cm}{2cm}{2cm}
    \nicsheader{\tiny\mono{\bs nicssmallskips}}
    \nicssmallskips
    \nicsitem{point 1}
    \nicsitem{point 2}
    \nicsitem{point 3}
    \nicsitem{point 4}
  \end{nicscolumn*}
  \begin{nicscolumn*}{7cm}{2cm}{2cm}
    \nicsheader{\tiny\mono{\bs nicsmedskips}}
    \nicsmedskips
    \nicsitem{point 1}
    \nicsitem{point 2}
    \nicsitem{point 3}
    \nicsitem{point 4}
  \end{nicscolumn*}
  \begin{nicscolumn*}{10cm}{2cm}{2cm}
    \nicsheader{\tiny\mono{\bs nicsbigskips}}
    \nicsbigskips
    \nicsitem{point 1}
    \nicsitem{point 2}
    \nicsitem{point 3}
    \nicsitem{point 4}
  \end{nicscolumn*}
\end{slide}

\begin{slide}{Colors}{}
  \begin{nicscolumn}
    \nicsitem{\textcolor{FireBrick}{hello colors}}
    \nicsitem{\textcolor{Coral}{you are all beautiful}}
    \nicsitem{\textcolor{ForestGreen}{yes, all of you}}

    \nicsitem{\SMALL\url{https://developer.mozilla.org/en-US/docs/Web/CSS/color_value\#Color_keywords}}
    \nicsitem{\href{http://www.yellowbearjourneys.com/color_themes/color_closest.html}{\Small \bold{www.yellowbearjourneys.com/color_themes/color_closest.html}}}

    \definecolor{customgreen}{HTML}{00D200}
    \nicsitem{\textcolor{customgreen}{It's ok to use custom colornames}}

    \nicsitem{We have a special style for cmdline: \nicscmdline{$ ls -l}}
  \end{nicscolumn}
\end{slide}

\nicsgrid1

\begin{slide}{Font sizes}{}
  \begin{nicscolumn*}{5cm}{0cm}{10cm}
    \def\example#1{#1\nicsitem{\string#1}}
    \example{\TINY}
    \example{\Tiny}
    \example{\tiny}
    \example{\SMALL}
    \example{\Small}
    \example{\small}
    \example{\normalsize}
    \example{\large}
    \example{\Large}
    \example{\LARGE}
    \example{\huge}
    \example{\Huge}
    \example{\HUGE}
  \end{nicscolumn*}
\end{slide}

\begin{slide}{Terminals}{We have an escape character to \LaTeX, so you can do fancy stuff}
  \begin{nicscolumn}
    \nicsmath
    \begin{nicsterm}
      59676@15:09 errge@brooks:~/work/nics-hello/demo $ ls -l
      total 988
      §\textcolor{red}{d}§rwxr-xr-x 1 errge errge     138 Jun 13 15:09 build
      drwxr-xr-x 1 errge errge      36 Jun 13 14:59 images
      -rw-r--r-- 1 errge errge      63 Jun 13 13:47 Makefile
      -rw-r--r-- 1 errge errge 1001300 Jun 13 15:09 slides.pdf
      -rw-r--r-- 1 errge errge    3880 Jun 13 15:09 slides.tex
      59677@15:09 errge@brooks:~/work/nics-hello/demo $ §\nicstermcursor\ $e^{i\pi}$
    \end{nicsterm}
  \end{nicscolumn}
\end{slide}

\begin{slide}{Code}{Without syntax highlighting}
  \begin{nicscolumn}
    \nicsitem{It's the same as terminals, but with light color scheme}
    \nicsitem{Terminals and codes (and other parts too) can also be zoomed}
    \nicszoom[4cm]{
      \begin{nicscode}
        int main() {
          return 0;
        }
      \end{nicscode}
    }
    \nicszoomnoautocenter1
    \nicszoom[3cm]{
      \begin{nicscode}
        int main() {
          return 0;
        }
      \end{nicscode}
    }
  \end{nicscolumn}
  \begin{nicscolumn*}{7cm}{5.5cm}{2cm}
    \nicszoom{\nicsmultiline{zooming \\ defaults \\ to \\ horizontal \\ size}}
  \end{nicscolumn*}
\end{slide}

\begin{slide}{Externalization}{For TikZ}
  \begin{nicscolumn}
    \nicsexterntemplate{tikz}
    \begin{nicsextern}{nics-is-awesome}
      \begin{tikzpicture}
        \node (left)  [ circle, draw = black, minimum width = 4cm, label=left:{\nicsmultiline{has \\ source \\ code}} ] {};
        \node (right) [ circle, draw = black, minimum width = 4cm, label=right:{\nicsmultiline{looks \\ nice}}, right=-1.5cm of left ] {};
        \node at ([xshift=-5mm] left.center) { RST/MD };
        \node at ([xshift=7mm] right.center) { PowerPoint };
        \node at ($ (left)!0.5!(right) $) [text = Tomato ] { nics };
      \end{tikzpicture}
    \end{nicsextern}
  \end{nicscolumn}
\end{slide}

\section{Externalization}

\begin{slide}{Externalization}{For slow to import images}
  \begin{nicscolumn}[3cm]
    \begin{nicsextern}{}
      \nicsincludepic{images/tux}
    \end{nicsextern}
  \end{nicscolumn}
\end{slide}

\begin{slide}{Externalization}{For source code highglighting}
  \begin{nicscolumn}
    \nicsitem{This is our killer-app compared to Google Slides/Powerpoint}
    \nicsexterncode{go}
    \begin{nicsextern}[height=5.5cm]{}
      package main

      func main() {
        b := string(96)
        print(p, b, p, b)
      }

      const p = `package main

      func main() {
        b := string(96)
        print(p, b, p, b)
      }

      const p = `
    \end{nicsextern}
  \end{nicscolumn}
\end{slide}

\end{document}
